\chapter{Conclusions}\label{chap:conclusions}

\epigraph{At terrestrial temperatures matter has complex properties which are likely to prove most difficult to unravel; but it is reasonable to hope that in the not too distant future we shall be competent to understand so simple a thing as a star.}{-- Arthur Eddington (1926)\footnotemark}

\footnotetext{\emph{The Internal Constitution of Stars}, Cambridge. (1926).}

In the earlier chapters, we discovered new insights into \gls{AGB} evolution and nucleosynthesis and its implications for measured chemical abundances. The main findings are outlined below:

In Chapter \ref{chap:shingleskarakas2013}, we asked if neutron-capture reactions in \gls{AGB} stars are the cause of the low sulphur abundances in planetary nebulae and post-AGB stars relative to the interstellar medium. Accounting for uncertainties in the size of the partial mixing zone (PMZ; which forms \iso{13}{C} pockets) and the rates of neutron-capture and neutron-producing reactions, our models failed to reproduce the observed levels of sulphur destruction. From this, we concluded that \gls{AGB} nucleosynthesis is not the cause of the sulphur anomaly. While addressing this question, we also discovered a new technique to constrain the size of the partial mixing zone that forms \iso{13}{C} pockets. Our constraint follows from the requirement that the AGB final surface abundances lie within the region spanned by planetary nebulae in the argon versus neon plane. This constrains the PMZ size to less than $5 \times 10^{-3}$ \Msun in our 3 \Msun model at a metallicity slightly below solar.

In Chapter \ref{chap:shinglesetal2014}, we studied the \sprocess enrichment of the globular clusters M4 and M22, which are examples of inter- and intra-cluster variation, respectively. Using a basic closed-box chemical evolution code, we predicted the relative increases of light-$s$ (Y, Zr) and heavy-$s$ (Ba, La, Ce) elements using stellar yields from the most likely candidate polluters -- rotating massive stars and AGB stars. We found that rotating massive stars alone do not explain the pattern of abundance variations, and that a contribution from both intermediate-mass AGB stars with a \iso{22}{Ne} neutron source and low-mass stars AGB with \iso{13}{C} pockets are required to explain the abundance variations. We also derived minimum enrichment timescales from the lowest mass (and longest-lived) stellar models in our best-fitting enrichment scenarios. Although this value depends on which assumptions are made about the partial-mixing zone, our estimate of 240 to 360 Myr for M22 is consistent with the literature value of a 300 Myr upper limit derived from isochrone fitting of the two stellar groups.

In Chapter \ref{chap:shinglesetal2015}, we explored the consequences of He-rich initial compositions for the stellar evolution and nucleosynthesis of intermediate-mass AGB stars. We found that the stellar yields of \sprocess elements were substantially lower in He-rich models, largely as a result of less intershell material being mixed into the envelope. We also found that envelope burning takes place at lower masses He-enhancement. The higher temperatures at the base of the convective envelope also suggest that \sprocess production by \iso{13}{C} pockets could be restricted to lower initial masses for higher helium abundances. Overall, our results demonstrate the importance of using models of the appropriate helium abundance when assembling sets of stellar yields for chemical evolution studies.

These studies demonstrate both the utility of stellar nucleosynthesis models for interpreting chemical abundances, as well as their limitations due to the current uncertainties in stellar physics.

A central theme in this thesis has been the uncertainties related to \iso{13}{C}-pocket formation, which is crucial for understanding the nucleosynthesis of heavy elements in low-mass stars. Despite its importance, we lack an understanding of \iso{13}{C}-pocket formation from first principles. A variety of constraints have been discovered from measured abundances (including the upper limit to the mixing depth discussed in Chapter \ref{chap:shingleskarakas2013}), but the physical mechanism that mixes protons into the He-rich core has not been determined conclusively. Currently there are several plausible mechanisms capable of producing \iso{13}{C} pockets that appear to be consistent with the present observational constraints. The formation mechanism could be an example of the more general uncertainty over modelling convection in one dimension, as convective overshooting has been shown to produce adequate \iso{13}{C} pockets for certain values of the overshoot parameter \citep{Herwig:2000ua,Cristallo:2009kn}. Other candidates for the physical process responsible are rotational mixing \citep{Langer:1999tj,Herwig:2001vb}, gravity-wave driven mixing \citep{Denissenkov:2003gx}, and semiconvection \citep{Iben:1982cv,Hollowell:1989bd}.

With a future physical understanding of \iso{13}{C} pocket formation, we will be able understand how the efficiency of the \iso{13}{C} neutron source varies as a function of the initial chemical composition and the initial stellar mass. This would improve the accuracy of stellar yields of \sprocess elements for chemical evolution studies.

The uncertainty of the mass-loss rate on the AGB was mentioned in Chapter \ref{chap:shinglesetal2015}. The differences in the evolution, nucleosynthesis, and stellar yields of our models compared to those of \citet{Straniero:2014jk} and \citet{Ventura:2009ha} are partly due to the use of different prescriptions for the mass-loss rate. Changes to the mass-loss rates alter the total number of thermal pulses, which affects the the number of dredge-up episodes and the nucleosynthesis that occurs in convective pulses (including via the \sprocess). The mass-loss rate also affects the duration and efficiency of hot bottom burning \citep{Ventura:2005ic}. For these reasons, our current uncertainty in the AGB mass-loss rate propagates to uncertainty in stellar yield predictions \citep{Stancliffe:2007er}.

\section{Future Directions}  %We can be done now that I've completed this work? What are the possible applications?
The general technique of using \sprocess element ratios to determine enrichment timescales demonstrated in Chapter \ref{chap:shinglesetal2014} could be further applied to other globular clusters with internal \sprocess variation such as M2 \citep{Yong:2014dq} and NGC 5286 \citep{Marino:2015ed}. The similarity (or variation) of the derived enrichment timescales could provide clues about the formation process of \sprocess-anomalous in general.

The enrichment timescales derived from chemical evolution models should in principle agree with age spreads derived by fitting isochrones to photometry. However, photometrically-inferred ages currently have very large uncertainties, and tend to be reliable only as upper limits. Our prediction of a 240--360 Myr enrichment timescale for M22 in Chapter \ref{chap:shinglesetal2014} is larger than the $150 \pm 50$ Myr estimate of \citet{Straniero:2014jk} mostly because their models include \iso{13}{C} pockets at higher initial masses. Unfortunately, both models are consistent with the $\sim$300 Myr age difference derived from isochrone fitting \citep{Marino:2012db,Joo:2013dr}. Assuming that the gas-cooling timescale is not a significant factor, a future measurement of the age spread in M22 from photometry that is more precise but much lower could potentially discriminate between the two models and would favour the formation of \iso{13}{C} pockets in more massive AGB stars.

Other analysis of chemical abundance studies (in e.g., post-AGB stars) could be a continuing source of progress on \iso{13}{C}-pocket formation. The theoretical route of using three-dimensional hydrodynamical simulations of TDU episodes will require either dramatically faster computers, new simulation codes with substantial performance optimisations and simplifications, or likely both.

%Existing work on planetary nebulae \citep{Miszalski:2013gi}, and presolar grains \citep{Liu:2014jh,Bisterzo:2014gk} has already placed some constraints on the size of \iso{13}{C} pockets in particular environments.

With the new stellar yields described in Chapter \ref{chap:shinglesetal2015} (and tabulated in Appendix \ref{app:yields}), and those of \citet{Karakas:2014ja} for 1.7 and 2.36 \Msun, we have a complete set of AGB stellar yields with a He content of $Y=0.24,0.30,0.35,$ and $0.40$. New chemical evolution models of $\omega$ Centauri can now be constructed using yields from stellar models of the appropriate helium abundance. The comparison between the chemical evolution models using our He-rich yields and the abundances of $\omega$ Centauri stars would be an ideal way to test the validity of our yield predictions. If we have enough confidence in the validity of the model, the comparison may support or reject our prediction that the \iso{13}{C} neutron source is suppressed at lower masses with increasing He content.

We currently do not know how our predictions for the stellar yields of He-rich stars would be altered with alternative modelling assumptions and different stellar evolution codes. Of critical importance for the stellar yield predictions is the quantity of intershell material that is transported to the surface by TDU. The finding of KMN14 and Chapter \ref{chap:shinglesetal2015} of this thesis that the total mass dredged-up by TDU is significantly reduced by He-enhancement should be investigated with other stellar evolution codes, which already predict much less efficient dredge-up for models with primordial He abundance \citep[see e.g.,][]{Mowlavi:1999ul,Lugaro:2003ew,Lugaro:2012ht}.

