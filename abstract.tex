\section*{Abstract}

Elements heavier than iron are almost entirely produced in stars through neutron captures and radioactive decays. Of these heavy elements, roughly half are produced by the \gls{sprocess}, which takes place under extended exposure to low neutron densities. Most of the \gls{sprocess} production occurs in stars with initial masses between roughly 0.8 and 8 \Msun, which evolve through the \gls{AGB} phase.

This thesis explores several topics related to \gls{AGB} stars and the \gls{sprocess}, with a focus on comparing theoretical models to observations in the literature on planetary nebulae, post-AGB stars, and globular cluster stars. A recurring theme is the uncertainty of \iso{13}{C}-pocket formation, which is crucial for building accurate models of \sprocess nucleosynthesis.

We first investigated whether neutron-capture reactions in \gls{AGB} stars are the cause of the low sulphur abundances in planetary nebulae and post-AGB stars relative to the interstellar medium. Accounting for uncertainties in the size of the partial mixing zone that forms \iso{13}{C} pockets and the rates of neutron-capture and neutron-producing reactions, our models failed to reproduce the observed levels of sulphur destruction. From this, we concluded that \gls{AGB} nucleosynthesis is not the cause of the sulphur anomaly. We also discovered a new method to constrain the extent of the partial mixing zone using neon abundances in planetary nebulae.

We next aimed to discover the stellar sites of the \gls{sprocess} enrichment in globular clusters that have inter- and intra-cluster variation, with the examples of M4 (relative to M5) and M22, respectively. Using a new chemical evolution code developed by the candidate, we tested models with stellar yields from rotating massive stars and \gls{AGB} stars. We compared our model predictions for the production of \sprocess elements with abundances from s-poor and s-rich populations. We found that rotating massive stars alone do not explain the pattern of abundance variations in either cluster, and that a contribution from \gls{AGB} stars with \iso{13}{C} pockets is required. We derived a minimum enrichment timescale from our best-fitting chemical evolution models and, although the value depends on the assumptions made about the formation of \iso{13}{C} pockets, our estimate of 240--360 Myr for M22 is consistent with the upper limit of 300 Myr inferred by isochrone fitting.

Lastly, there is accumulating evidence that some stars (e.g., in $\omega$ Centauri) have been born with helium mass fractions as high as 40\%. This motivated us to explore the impact of helium-rich abundances on the evolution and nucleosynthesis of intermediate-mass (3--6 \Msun) \gls{AGB} models. We found that the stellar yields of \gls{sprocess} elements are substantially lower in He-rich models, largely as a result of less intershell material being mixed into the envelope. We also found evidence that high He abundances could restrict the \sprocess production by \iso{13}{C} pockets to stars with lower initial masses.

\glsresetall