\section*{Acknowledgements}
I am thankful for the knowledge, techniques, and culture of science that were developed before my time. Growing up, I had learnt a fair number of scientific facts, but I did so without really believing that these ideas could describe the workings of daily life outside of specialised laboratory experiments. It was far too long before I understood that the known laws of physics are either consistent with literally every aspect of human experience, or we would have evidence that they are wrong and they need to be revised. For me, science then became more important as a way to understand the world while avoiding flawed or irrational thinking.

I am grateful to my primary supervisor, Amanda Karakas for her expert guidance of my research programme. Having observed how heavily other students' progress has been influenced by their supervisors and chosen research topics, I feel very fortunate to have been supervised by Amanda on the topic of stellar evolution and nucleosynthesis. Although she encouraged me to work independently where possible, Amanda has also been very generous with her time and knowledge throughout my PhD. Her excellent teaching skills clearly have applications beyond stellar astrophysics, as she has given me a solid foundation in workplace soft skills such as cocktail making. I'm sure this knowledge will continue to serve me well as I make the transition to postdoctoral life.

Thank you to the rest of my advisory panel, Gary Da Costa, David Yong, Richard Stancliffe, and John Lattanzio. Each of them has given their time to provide me with useful feedback and advice. Thank you also to my collaborators (especially Carolyn Doherty) who have helped me to improve the quality and scientific rigour of my papers.

Thank you to everyone at Stromlo for contributing to a friendly work environment. I would list people by name, but I would inevitably forget someone important, so I will just single out my office mates: Christopher Nolan and Chris Owen. Thanks for waiving the entrance requirements and letting me join the `Chris' office. Thanks to my climbing buddies for helping me to get more exercise and spend less time sitting down. Thanks to everyone at cosmology lunch and supernova tea for welcoming me into the field of supernovae and helping me to prepare for my job interview (and now postdoc) at Queen's University Belfast.

Thank you to my immediate family, Mum, Dad, my sister Tegan, and my brother, Josh. Dr. Joshua Shingles gave me no excuses not to succeed in my doctorate after he completed his while simultaneously working full-time and raising four children with his wife Natalie.

Thank you to Louise for being with me during the best and worst times of our PhD years. I'll remember the great times we spent having tea after work (at exactly 5pm!), spending nights eating in and watching TV at our apartment, having chicken schnitzels and microwaved vegetables ridiculously often, going out for lunch or coffee on the weekends, and our holiday trips together (especially Christmases!). Thank you for introducing me to much of the culture, accents, phrases, and retail stores of the UK ahead of my move to Belfast. I'm reminded of you when I go to a Caff\'e Nero, Boots, Tesco, or Pizza Express, and whenever I have no idea whether a British person is joking or not.